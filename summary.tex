\documentclass[11pt,landscape,a4paper,fleqn]{article}
\usepackage[utf8]{inputenc}
\usepackage[]{babel}
\usepackage{tikz}
\usetikzlibrary{shapes,positioning,arrows,fit,calc,graphs,graphs.standard}
\usepackage{multicol}
\usepackage{wrapfig}
\usepackage[top=3mm,bottom=4mm,left=3mm,right=3mm]{geometry}
\usepackage[framemethod=tikz]{mdframed}
\usepackage{microtype}
\usepackage{mathtools}
\usepackage{ccicons}
\usepackage{hyperref}
\usepackage{lmodern}
\usepackage{xcolor}
\usepackage{soul}
\usepackage{amssymb}
\usepackage[neveradjust]{paralist}
\usepackage[shortlabels]{enumitem}
\usepackage{bbm}
\usepackage{listings}
\usepackage{libertine}
\usepackage[libertine]{newtxmath}
% \usepackage{ETbb}
% \usepackage[sc]{mathpazo}
\usepackage{algpseudocode}
\usepackage{inconsolata}
\usepackage{physics}

\providetoggle{showextratext}
\settoggle{showextratext}{false}

\setlength{\columnsep}{2mm}

\setlist{topsep=0pt, leftmargin=*, noitemsep, topsep=0pt,parsep=0pt,partopsep=0pt}

\newcommand{\extratext}[1]{\iftoggle{showextratext}{#1}{}}

\newcommand*{\tran}{^{\mathsf{T}}} % (DIN) EN ISO 80000-2:2013
\newcommand{\kl}[2]{D_{\mathrm{KL}}(#1\lVert#2)}

\DeclareMathOperator{\E}{\mathbb{E}}

\setdefaultleftmargin{0.5cm}{}{}{}{}{}

\let\bar\overline

\definecolor{myblue}{cmyk}{1,.72,0,.38}
\definecolor{myorange}{cmyk}{0,0.5,1,0}
\definecolor{myorange2}{cmyk}{0,0.8,0.8,0}
\definecolor{darkgreen}{cmyk}{0.97,0,1,0.57}

\pgfdeclarelayer{background}
\pgfsetlayers{background,main}

%\everymath\expandafter{\the\everymath \color{myblue}}
%\everydisplay\expandafter{\the\everydisplay \color{myblue}}

\renewcommand{\baselinestretch}{.8}
\pagestyle{empty}

\global\mdfdefinestyle{header}{%
linecolor=gray,linewidth=1pt,%
leftmargin=0mm,rightmargin=0mm,skipbelow=0mm,skipabove=0mm,
}

\newcommand{\header}{
\begin{mdframed}[style=header]
\footnotesize
\sffamily
Cheat sheet\\
Yannick Merkli,~page~\thepage~of~2
\end{mdframed}
}

\makeatletter
\renewcommand{\section}{\@startsection{section}{1}{0mm}%
                                {0pt}%
                                {0.5pt}%x
                                {\color{myorange}\sffamily\small\bfseries}}
\renewcommand{\subsection}{\@startsection{subsection}{1}{0mm}%
                                {0pt}%
                                {0.1pt}%x
                            	{\color{myorange2}\sffamily\small}}


\makeatother
\setlength{\parindent}{0pt}

\newcommand{\mhl}[1]{\setlength{\fboxsep}{0pt}\colorbox{yellow}{#1}}
\newcommand{\red}[1]{\textcolor{red}{#1}}
\newcommand\iid{\stackrel{\mathclap{\normalfont\mbox{\tiny{iid}}}}{=}}
\lstset{
	basicstyle=\small,
	mathescape
}

\begin{document}
	
% \section*{Disclaimer}

\setlength{\columnseprule}{0.1pt}
\begin{multicols*}{4}
    
\section{Generative modelling}

Learn $p_{\mathrm{model}} \approx p_{\mathrm{data}}$, sample from $p_{\mathrm{model}}$.

\begin{itemize}[leftmargin=0.5em]
    \item Explicit density:
    \begin{itemize}[leftmargin=0.3em]
        \item Approximate:
        \begin{itemize}[leftmargin=0.0em]
            \item Variational: VAE, Diffusion
            \item \textcolor{gray}{Markov Chain: Boltzmann machine}
        \end{itemize}
        \item Tractable:
        \begin{itemize}[leftmargin=0.0em]
            \item Autoregressive: FVSBN/NADE/MADE, Pixel\textsc{(c/r)nn}, WaveNet/\textsc{tcn}, Autor. Transf., 
            \item Normalizing Flows
        \end{itemize}
    \end{itemize}
    \item Implicit density:
    \begin{itemize}[leftmargin=0.3em]
        \item Direct: Generative Adversarial Networks
        \item \textcolor{gray}{MC: Generative Stochastic Networks}
    \end{itemize}
\end{itemize}

Autoencoder: $X \textcolor{blue}{\to} Z \textcolor{red}{\to} X$, $\textcolor{red}{g} \circ \textcolor{blue}{f} \approx \mathrm{id}$,
$f$ and $g$ are NNs. Optimal linear autoencoder is PCA.

Undercomplete: $|Z| < |X|$, else overcomplete.
Overcomp. is for denoising, inpainting.

Latent space should be continuious and interpolable.
Autoencoder spaces are neither,
so they are only good for reconstruction.

\section{Variational AutoEncoder (VAE)}

Sample $z$ from prior $p_\theta(z)$, to decode use conditional $p_\theta(x \mid z)$ defined by a NN.

$\kl{P}{Q} \coloneqq \int_x p(x) \log \frac{p(x)}{q(x)} \dd x$: KL divergence,
measure similarity of prob. distr.

$\kl{P}{Q} \neq \kl{Q}{P}, \kl{P}{Q} \geq 0$

Likelihood $p_\theta(x) = \int_z p_\theta(x \mid z) p_\theta(z) \dd z$ is hard to maximize,
let encoder NN define $q_\phi(z \mid x)$,
\qquad\qquad
$\log p_\theta(x^{i}) = \textcolor{orange}{\E_z\left[\log p_\theta(x^{i} \mid z)\right]}
- \textcolor{purple}{\kl{q_\phi(z \mid x^{i})}{p_\theta(z)}} + \textcolor{red}{\kl{q_\phi(z \mid x^{i})}{p_\theta(z \mid x^{i})}}$.
\textcolor{red}{Red} is intractable, use $\geq 0$ to ignore it;
\textcolor{orange}{Orange} is reconstruction loss, clusters similar samples;
\textcolor{purple}{Purple} makes posterior close to prior, adds cont. and interp.
$\mathrm{\textcolor{orange}{Orange}} - \mathrm{\textcolor{purple}{Purple}}$ is \textbf{ELBO}, maximize it.

$x \xrightarrow{\mathrm{\makebox[0pt]{\scriptsize \textcolor{blue}{enc}}}} \mu_{z \mid x}, \Sigma_{z \mid x} \xrightarrow{\mathrm{sample}} z \xrightarrow{\mathrm{\makebox[0pt]{\scriptsize \textcolor{red}{dec}}}} \mu_{x \mid z}, \Sigma_{x \mid z}  \xrightarrow{\mathrm{sample}} \hat{x}$

Backprop through sample by reparametr.: $z = \mu + \sigma \epsilon$.
For inference, use $\mu$ directly.

Disentanglement: features should correspond to distinct factors of variation.
Can be done with semi-supervised learning by making $z$ conditionally independent of given features $y$.

\subsection{$\beta$-VAE}

Disentangle by
$\max_{\theta, \phi} \E_x\left[\E_{z \sim q_\phi} \log p_\theta(x \mid z)\right]$ s.t.
$\kl{q_\phi(z \mid x)}{p_\theta(z)} < \delta$, with KKT: $\max \textcolor{orange}{\mathrm{Orange}} - \beta\textcolor{purple}{\mathrm{Purple}}$.

\end{multicols*}
\end{document}